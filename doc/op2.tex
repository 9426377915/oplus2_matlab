\documentclass[12pt]{article}

\usepackage[colorlinks]{hyperref}

\topmargin 0in
\textheight 8.75in
\textwidth  5.75in

\parskip 5pt

\newenvironment{routine}[2]
{\vspace{.25in}{\noindent\bf\hspace{0pt} #1}{\\ \noindent #2}
\begin{list}{}{
\renewcommand{\makelabel}[1]{{\bf ##1} \hfil} 
\itemsep 0pt plus 1pt minus 1pt
\leftmargin  1.2in
\rightmargin 0.0in
\labelwidth  1.1in
\itemindent  0.0in
\listparindent  0.0in
\labelsep    0.05in}
}{\end{list}}
%

\usepackage{algorithm}
\usepackage{algorithmic}

%\newcommand{\theHalgorithm}{\arabic{algorithm}}

\newenvironment{alg}[2]
{
\vspace{-0.5cm}
\begin{flushleft}
\begin{minipage}[c]{0.8\linewidth}
\begin{algorithm}[H]
\caption{#1}
\label{#2}
\begin{algorithmic}
}
{
\end{algorithmic}
\end{algorithm}
\end{minipage}
\end{flushleft}
}


\begin{document}

\title{OPlus2 for many-core platforms}
\author{Mike~Giles}
\date{April 7, 2010\\[0.1in] (updated to rename ``pointers'' as ``mappings'')}

\maketitle

\begin{abstract}
This document lays out the plans for a new open source OPlus2 
library for unstructured grid parallel execution on NVIDIA GPUs
and other many-core hardware platforms.

This document will be all-encompassing to begin with.  Later, 
it will make sense to split it into different documents 
(e.g. Users Guide; Developers Guide; Algorithms for 
partitioning, colouring and halo construction) but for now 
I think it is simplest to have it all together in one place.
\end{abstract}


\section{Motivation}

The original OPlus library was developed more than 10 years ago
for distributed memory parallel execution of algorithms based
on unstructured grids. It was originally developed in PVM by
Paul Crumpton and myself, but more recently Nick Hills ported
it to MPI and made numerous internal changes (such as parallel
partitioning) which greatly improved its performance.

The strengths and weaknesses of the existing OPlus are discussed 
below, but the main motivation for a new version is the emergence 
of multicore architectures.  Not only are Intel, AMD, and others
all pursuing a multicore CPU strategy, with up to 6 cores today 
but maybe 16 cores (and up to 64 threads?) within a few years, 
but also NVIDIA and ATI (now part of AMD) have graphics cards 
with huge numbers of cores.  The current NVIDIA GTX280 GPU, 
for example, has 240 cores, each of which is as capable as an 
Intel Core 2 Duo core in single precision.

The objective of OPlus2 is to provide a library which will enable
efficient execution on these new architectures at the lowest 
level, while retaining the ability to perform distributed memory 
parallelisation on the highest level.


\newpage

\section{OPlus}

After more than 10 years of use, it is a good time to review the 
strengths and weaknesses of the existing OPlus library.

Strengths:
\begin{itemize}
\item
general distributed set/mapping representation has worked well
\item
single application source code leading to both sequential and 
parallel executables
\item
fairly clean definition of parallel loops
\item
good scalability on big problems with up to $10^8$ grid nodes
on 100+ processors (partly due to the original design 
but largely due to Nick Hills' improvements)
\item
works well under both Unix/Linux and Windows operating systems
\end{itemize}

Weaknesses:
\begin{itemize}
\item
file I/O is poor, not parallel
\item
{\tt OP\_ACCESS} declaration of data dependencies in parallel 
loops is tedious, and errors can be hard to debug
\item
only FORTRAN is supported, and it is only used by the HYDRA CFD code
\end{itemize}


The natural goal with OPlus2 is to retain the strengths, address the 
weaknesses, and add the capability to efficiently exploit new many-core 
architectures.

\subsection*{References}

1) P.I. Crumpton and M.B. Giles. `Multigrid aircraft computations using 
the OPlus parallel library'. in  Parallel Computational Fluid Dynamics: 
Implementations and Results Using Parallel Computers, 339-346. A. Ecer, 
J. Periaux, N. Satofuka, and S. Taylor, editors, North-Holland, 1996.  
\href{http://people.maths.ox.ac.uk/~gilesm/files/para_cfd95.pdf}{(PDF)}

2) D.A. Burgess, P.I. Crumpton, and M.B. Giles. `A parallel framework 
for unstructured grid solvers'. In Computational Fluid Dynamics '94: 
Proceedings of the Second European Computational Fluid Dynamics 
Conference, pages 391-396. S. Wagner, E.H. Hirschel, J. Periaux, and 
R. Piva, editors. John Wiley and Sons, 1994.  
\href{http://people.maths.ox.ac.uk/~gilesm/files/NA-95-20.pdf}{(PDF)}

\newpage

\section{Key assumptions and requirements}

The key assumptions are the same as for OPlus: that the 
application involves a number of static sets (e.g.~nodes, edges, 
faces, cells) connected by mappings, and each step in the algorithm 
involves an operation applied to all elements of a set (with at
most one level of mapping indirection to data associated with another
set) and the result is independent of the order in which the elements 
are processed (e.g.~Jacobi iteration is allowed but not the standard 
Gauss-Seidel).

The requirements are:
\begin{itemize}
\item
single application source code leading to efficient execution on 
\begin{itemize}
\item
multicore CPUs using OpenMP and/or SSE/AVX instructions
\item
GPUs using CUDA or OpenCL
\end{itemize}

\item
very good scalability up to very large problem sizes using MPI 
to exploit multiple GPUs and/or multicore CPUs within a cluster

\item 
support for both C/C++ and FORTRAN

\item
support for Unix/Linux and Windows operating systems

\item
options for extensive error-checking

\item
compatibility with HYDRA's use of automatic differentiation 
for generating linear and adjoint code

\end{itemize}

With OPlus, the user code was compiled and then linked to either a 
sequential single-thread library or a parallel library (based on MPI
message passing).  There was no use of program transformation, partly 
because it wasn't needed and partly because we had no knowledge of 
such tools.

With OPlus2, the efficient support of multicore CPUs as well as GPUs 
will require the use of program transformation tools, with the expertise 
being supplied by Paul Kelly's group at Imperial College.
However, I think it is still desirable to have a non-efficient single 
thread CPU implementation which does not require program transformation; 
this can be used for application development and debugging and should 
include the extensive error-checking tools mentioned above.


\newpage

\section{Outline workplan}

The development of OPlus2 will proceed in two phases, with perhaps some 
overlap between the two.
\begin{itemize}
\item
Phase 1 will develop OPlus2 for use on a single GPU or a single 
shared-memory system with one or more multicore CPUs (using OpenMP and 
SSE/AVX instructions).
\item
Phase 2 will extend this by adding on top an MPI layer which is very
similar to the existing OPlus implementation.  This will enable
distributed-memory computations on a cluster of nodes, each with 
one or more GPUs and/or multi-core CPUs.
\end{itemize}

The partitioning and parallelisation strategies required by the two
phases will be similar in some ways, but will also be different because
of differences in key hardware aspects.  For example, typical cluster
compute nodes have GBs of shared memory, whereas each ``multiprocessor'' 
in an NVIDIA GPU currently has only 16kB of shared memory.  As explained 
later, this leads to a different approach in handling data dependency 
constraints.


The phase 1 workplan is as follows (IC = Imperial College post-doc):
\begin{itemize}
\item
single-thread CPU version of test code written in C,
including non-efficient CPU library (Mike, Sept 09)

\item
CUDA version of test code written in C (Mike, Oct/Nov 09)

\item
document describing program transformation requirements for 
C $\rightarrow$ CUDA (Mike/Paul, Dec 09)

\item
single-thread CPU version of test code written in FORTRAN,
including non-efficient CPU library (Mike, Jan 10)

\item
C and FORTRAN versions of 2D airfoil test code (Mike, Mar 10)

\item
program transformation for C $\rightarrow$ CUDA (IC, June 10)

\item
OpenMP/SSE version of original test code (Mike, June 10)

\item
program transformation for FORTRAN $\rightarrow$ CUDA 
(IC, Sept 10)

\item
test program transformation tools on 2D airfoil test code 
(Mike, Sept 10)

\item
program transformation for C $\rightarrow$ OpenMP/SSE parallel 
loops (IC, Dec 10)

\item
OpenCL version of original test code (Mike, Dec 10)

\item
``lint''-style syntactic analysis and error-checking for 
parallel loop kernels (IC, Mar 11)

\item
program transformation tool to aid porting of HYDRA OPlus
calls to OPlus2 (IC, June 11)

\item
begin phase 2 (Mike, July--Sept 11)

\item
program transformation for C, FORTRAN $\rightarrow$ OpenCL 
(IC, Sept 11)
\end{itemize}


%This timetable gives Paul's postdoc 3 months (Jan-Mar 09) to 
%familiarise themselves with OPlus and Rose.  Also, the OpenCL
%tasks could be tackled before OpenMP/SSE, depending on the
%relative priorities set by RR.

At the end of the first phase we will have a viable library 
and program transformation tools for execution on a single host 
with one or more NVIDIA GPUs.  

The porting of HYDRA will overlap with the final parts of phase 1 
and the early parts of phase 2, with the three main steps being:
\begin{itemize}
\item
convert remaining parallel loops into kernel format
(this can be done at any time, independent of work on phase 1);

\item
convert OPlus parallel loops into OPlus2 parallel loops,
aided by the tool developed by Imperial;

\item
convert file I/O to the new parallel file I/O agreed for phase 2.
\end{itemize}

\noindent
The timing for phase 2 is a bit uncertain, but the key steps 
(some of which can be overlapped) are:
\begin{itemize}
\item
review existing OPlus algorithms and code (Mike, 3 months?)

\item
review HDF5 and MPI I/O and design OPlus2 file I/O (Leigh, 3 months?)

\item
review key implementation issues and concerns (Nick, 3 months?)

\item
implement new file I/O (1 month?)

\item
implement phase 2 data partitioning and halo construction 
(3 months?)

\item
test on a conventional cluster (1 month?)

\item
optimise GPU partitioning to overlap MPI communication with GPU 
computation (1 month?)

\item
test on a GPU cluster (1 month?)

\item
test HYDRA on a GPU cluster (1 month?)

\item
develop performance monitoring tools to guide further development 
(1 month?)
\end{itemize}
The work in most of these items will be done jointly by the 
Imperial College postdoc and myself.

\newpage

\section{Data access and dependency conflicts}

Suppose that one wants to solve the sparse system of linear equations
\[
A\, u + r = 0
\]
using Jacobi iteration with
\[
u^{n+1} = u^n + \left( A\, u^n + r\right).
\]
$A$ can be stored in a sparse edge-based format 
\verb!A,p1,p2! in which each non-zero element has a corresponding ``edge'' 
\verb!e! with
\verb!A[e]!$\equiv A_{i,j}$, 
\verb!p1[e]!$\equiv i$, 
\verb!p2[e]!$\equiv j$. 

Using this, the key steps in the algorithm can be implemented using the C
routines in Figure \ref{fig1}.  The access patterns for the arrays can be 
classified as
\begin{itemize}
\item ``r''  read-only
\item ``w''  write-only
\item ``b''  both read and write
\item ``i''  increment  (a special case of read/write)
\end{itemize}
In \verb!res!, the access for \verb!A, u, du! is ``r'', ``r'' and ``i'', 
respectively, while
in \verb!update!, the access for \verb!u, du, r! is ``i'', ``b'' and ``r'', 
respectively.


\begin{figure}[b!]
\begin{verbatim}


void res(int ne, int *p1, int *p2, float *A, float *u, float *du) {
   for (int e=0; e<nedge; e++)
      du[p1[e]] += A[e] * u[p2[e]];
}

void update(int nn, float *u, float *du, float *r) {
   for (int n=0; n<nn; n++) {
      u[n] += du[n] + r[n];
      du[n] = 0.0f;
   }
}
\end{verbatim}
\caption{C routines for simple Jacobi iteration}
\label{fig1}
\end{figure}


It is clear from looking at these routines that it makes no difference the 
order in which the edges are processed in \verb!res!, or the nodes are processed
in \verb!update!, and so these conform to the requirements for OPlus 
parallelisation.  The one potential problem is a potential data dependency
conflict in incrementing the array \verb!du! in \verb!res!.  Because of the
indirect addressing using \verb!p1!, it is possible that different edges may 
try to update the same element of \verb!du! ``at the same time''.  Unless the 
hardware supports atomic updates for floating point variables, this could lead
to incorrect results.

In OPlus, the parallelisation partitions the data so that each partition 
``owns'' some of the edges and some of the nodes.  Each partition ``executes''
the edges and nodes which affect the data it ``owns''.  Near partition 
boundaries, this may require one partition to execute an edge belonging to a
different partition.  In addition, if the execution of the edge requires the
use of data belonging to another partition, a copy of that data has to be made;
this is referred to as ``halo'' data in the distributed-computing community.

Within each partition, a single thread sequential execution model is used and 
so there is problem with the data dependency issue.  However, in more complex
applications where executing an edge may involve updates to more than one node, 
there is a redundancy/duplication issue, since the same edge may be executed 
by more than one partition. This is acceptable provided the partition sizes 
are large enough so that the number of halo edges and nodes is relatively small.

In OPlus2, the same strategy will be used in phase 2 for distributed computing
across multiple GPUs, but in phase 1 the same approach will not be acceptable
because the partition sizes to fit into each of the multiprocessors within a 
single GPU are so small that there will be a significant level of duplication.
We will instead use an alternative partition ``colouring'' strategy which
is described in a later section.

\newpage

\section{CUDA parallelisation strategy}


The first step is to reorder/renumber all sets to improve data locality,
so that neighbouring nodes or edges have indices, and therefore memory 
locations, which are close to each other.   There are lots of ways in which 
this might be done, but one simple effective way which we have used in the past 
is recursive coordinate bisection.  For a 3D dataset, one would first bisect
each set in the $x$-direction, then the $y$-direction, then the $z$-direction, 
and then repeat the sequence recursively to build up a binary tree whose leaves 
are then numbered sequentially.

The idea then is to parallelise loop operations by using blocks 
(mini-partitions) which are small enough to fit into the limited shared 
memory on each multiprocessor within an NVIDIA GPU.  For the edge 
operations in \verb!res! in Figure \ref{fig1}, the operations for 
each block would be
\begin{itemize}
\item
load in required values of \verb!u! into shared memory
\item
calculate increment
\item
add increment to shared memory array \verb!inc!
\item
add increment \verb!inc! to array \verb!du! in main graphics memory
\end{itemize}

The first step of loading \verb!u! into shared memory is performed because
the same values of \verb!u! are likely to be used by several edges, and
so doing this minimises the total data transfer requirements from the main 
graphics memory.
  
The second step is simple, and can be performed in parallel by all threads 
in the block (with just one edge per thread?) without any data conflicts.  
The potential data conflict is in the third step when the increments are 
combined.  This will be avoided by ``colouring'' the threads so that no 
two threads update the same variable at the same time.  This is a 
well-established technique for avoiding data conflicts in vector code.

(To minimise the number of colours within each thread ``warp'', it may
be best to reorder the threads within each block.  This would scramble 
up the identity mapping, used here for the array \verb!A!, so there would 
be some cost but I think there would be savings overall.)

The fourth step adds the combined increments to the data in the main 
graphics memory.  Here again there is a potential data conflict problem 
due to different blocks trying to update the same data.  This will be 
avoided by colouring the blocks so that no two blocks of the same colour 
update the same data.  The blocks would be executed one colour at a time, 
with a synchronisation between each colour.


\subsection{Execution plans}

In standard MPI computations, the partitioning is done once and the same
partitioning is used for stage of the computation.  This is because 
the cost of re-partitioning the data is excessive.

In contrast, between each stage of the computation on the GPU the data 
resides in the main graphics memory, and so the blocking for each loop 
calculation can be considered independently of the requirements of the
other parallel loops.

Based on ideas from FFTW, I plan to construct for each parallel loop
a ``plan'', which is a customised blocking of the execution on the GPU 
for that loop, making optimum use of the local shared memory on each 
multiprocessor considering in detail the memory requirements of the loop 
computation.  

The plan will also include the colouring of the blocks, and colouring 
of the threads within a block, to avoid data conflicts.  A typical CFD 
calculation may need 50-200 plans each of which may be executed more 
than once during a single timestep or multigrid iteration.

\subsection{Some rough numbers}

A big HYDRA CFD calculation will use almost all of the 4GB 
on a Tesla card. Given that the local shared-memory size is 16kB, 
this suggests that the data intensive parallel loops will have 
over 100,000 blocks. 
\footnote{On the one hand, this will be an underestimate 
since some of the data will be used by more than one block, 
but on the other hand it will be an overestimate 
because not all of the data will be needed for just one loop.}

10 colours might be needed to avoid data conflicts, suggesting 
up to 10,000 blocks per colour.  Given this number it doesn't 
matter if there are variations in the amount of compute per 
block, the important thing is to make sure each block uses as 
much shared memory as possible.

HYDRA needs up to 40 floats per grid node.  16kB corresponds
to 4000 floats which is equivalent to 100 nodes, which is 
roughly $5^3$.  This should be big enough to get a fair degree 
of re-use of nodal data within the block, maybe 50\% of
maximum.  The block may have 400 edges, with maybe 40 per 
colour so we need to use colour-locked increments, with thread 
synchronisation in between colours.


\newpage

\section{OP2 execution}


An OPlus2 calculation has three stages of execution, each with a 
number of steps.  In the description below, the steps in parentheses 
are the extra ones needed for the phase 2 MPI parallelisation:
\begin{enumerate}
\item
initialisation
 \begin{itemize}
 \item
 declare sets, pointers and datasets
 \item
 (partition sets and renumber pointers)
 \item
 (compute halos)
 \item
 (load and distribute datasets)
 \item
 renumber elements and pointers
 \item
 initialise the GPU, copy constants to its constant memory 
 and copy datasets to its global memory
 \end{itemize}

\item
parallel computation
 \begin{itemize}
 \item
 (exchange halo data as necessary)
 \item
 build new CUDA execution ``plan'', if necessary 
 \item
 execute ``plan''
 \end{itemize}


\item
termination
 \begin{itemize}
 \item
 copy datasets back to host
 \item
 terminate CUDA operations on the GPU
 \item
 (write datasets to disk)
 \end{itemize}
\end{enumerate}


\newpage

\section{Parallel loop syntax}

My initial idea is that the parallel loop should have the following syntax:

\begin{routine} {op\_par\_loop(kernel, set, label, \\
\hspace*{1.4in}  arg1, map1, index1, type1, dim1, access1, \\
\hspace*{1.4in}  arg2, map2, index2, type2, dim2, access2, \\ 
\hspace*{1.4in}  \ldots,\ \ \ \ldots,\ \ \ \ldots,\ \ \ \ldots,\ \ \ \ldots,\  \ \ \ldots,  \\
\hspace*{1.4in}  argN, mapN, indexN, typeN, dimN, accessN)}
{}%{This is a routine returning no value.}

\item[kernel]   name of the kernel function to be applied to each element
\item[label]    a label string for diagnostic purposes
\item[set]      set whose elements are to be processed 
\item[arg1]     first dataset
\item[map1]     pointer from main set to set on which {\bf arg1} is defined
\item[index1]   index of pointer to access {\bf arg1}
\item[type1]    type of dataset (e.g. ``float'', ``int'')
\item[dim11]    dimension of dataset
\item[access1]  information on how {\bf arg1} is used within kernel\\
                ``r'' = read-only\\
                ``w'' = write-only\\
                ``b'' = both read + write\\
                ``i'' = increment
\end{routine}


Here {\bf kernel} is a routine which process a single element in {\bf set}.
This will get converted by a pre-processor into a routine called by the 
CUDA kernel function.  The pre-processor will also take the specification 
of the arguments and turn this into the CUDA kernel function which loads 
in the data from the device main memory into the shared storage, then calls 
the converted {\bf kernel} for each element, then writes back the updated 
data to the device main memory.

Parallel loops of the above form will form the main bulk of the application 
code, after an initialisation stage in which the sets, the pointers between 
the sets and the data associated with the sets are all declared.

This syntax combines the functions of {\bf OP\_PAR\_LOOP} and 
{\bf OP\_ACCESS} in the current OPlus, and the specification of the 
single element {\bf kernel} function fits in well with the use of
automatic differentiation in HYDRA.

There is a fair amount of redundant information here, but the implementation
will check it is consistent; see discussion in Blog section.


\newpage

There are also routines to declare sets, mappings and datasets:

\begin{routine} {op\_decl\_set(elems, dim, coords, set, label, mapping, maplabel)}
{}%
\item[elems]    number of elements in the set
\item[dim]      dimension of the coordinate data (for renumbering)
\item[coords]   coordinate data
\item[set]      structure containing set info (set by routine)
\item[label]    a label string for diagnostic purposes
\item[mapping]  an identity mapping (needed for some parallel loops)
\item[maplabel] a label for the identity mapping
\end{routine}

\begin{routine} {op\_decl\_map(from\_set, to\_set, dim, mapdata, map, label)}
{}%
\item[from\_set] set from which it points
\item[to\_set]   set to which it points
\item[dim]       number of pointers per element
\item[mapdata]   input pointer data
\item[map]       structure containing map info (set by routine)
\item[label]     a label string for diagnostic purposes
\end{routine}

\begin{routine} {op\_decl\_dat(set, dim, type, data, dat, label)}
{}%
\item[set]       set with which the data is associated
\item[dim]       amount of data per element
\item[type]      type of data (e.g. ``float'', ``int'')
\item[data]      input data array
\item[map]       structure containing data info (set by routine)
\item[label]     a label string for diagnostic purposes
\end{routine}

\section{Execution plan construction}

New notes: 1/11/09

I've almost finished the coding for the execution plan construction so 
it's time to make some notes, mainly for my own benefit but also to help
anyone reading the code.

A call to {\bf op\_par\_loop} leads to a call to {\bf plan} which 
looks to see if there is an existing plan.  If there is, it returns 
its ID, otherwise it constructs a new plan and returns its ID.
There is a match to an existing plan if all of the arguments match.  
Note that this needs to be a run-time match; it is not something that 
can be decided at compile-time since the same call in the code can be
used for different levels within a multigrid solver, and each level 
requires its own execution plan.

In constructing a new plan, the first step is to see if the loop
involves any indirection.  If it does not, the execution plan and the
corresponding CUDA code is very simple.

When there is indirection, the construction of the plan involves the 
following steps:
\begin{itemize}
\item
identification of the datasets involving indirection, and the 
``increment'' subset of these which involve an {\bf OP\_INC} access 
producing a potential data conflict which requires coloring

\item
for each thread block:
  \begin{itemize}
  \item for each indirection dataset:
    \begin{itemize}
    \item assemble list of elements pointed to
    \item sort into ascending order and eleminate duplicates
    \item store away for use by CUDA routine
    \item re-number pointers accordingly and store them
    \end{itemize}
  \item color the threads (and perhaps re-order threads
        to group by color to minimise warp divergence?)
  \end{itemize}

\item
color the blocks (and perhaps re-order the blocks to group 
by color?) 
\end{itemize}

\newpage

\subsection{Coloring}

Coloring is potentially time-consuming, so an important ``trick'' here 
is to assign 32 colors at a time using bit operations.  I explain it here 
for thread coloring but the same approach is also used for block coloring.

The key to the implementation is an array with a single 32-bit unsigned 
integer for each indirectly addressed increment set element.  This is 
initialised to zero and then, as colors are assigned, the corresponding 
bits will be set to indicate the color of the elements accessing that 
indirectly addressed element.

The algorithm proceeds as follows:
\begin{itemize}
\item
loop over all elements which have not yet been colored, and for each one
  \begin{itemize}
  \item
  initialise mask variable to zero
  \item
  loop over all indirect increment set elements and ``add'' to the mask
  using the logical OR operation
  \item
  find first bit of mask which is \underline{not} set; this is the color 
  of the new element (provided there is a bit not set)
  \item
  reset the mask to have the appropriate color bit set
  \item
  loop back over all indirect increment set elements ``adding'' the mask
  usign the logical OR operation to set the appropriate the  color bit
  \end{itemize}
\item
if 32 colors weren't sufficient, reset the work array to zero and repeat 
the whole procedure for those which have not yet been colored
\end{itemize}


In most cases I expect that 32 colors will be sufficient, and so a 
single pass through each thread block will be all that is required 
for the thread coloring, and then a single pass through the entire 
set for the block coloring.

\subsection{Different access type for same dataset}

What should be done if different argument involve the same dataset 
with different access types?

\begin{itemize}
\item
treat each access type as giving a different indirect dataset 
for the purposes of the execution plan construction
\item
requires there to be \underline{no} data conflict between these
different sets; must be explained carefully in documentation,
and checked in code which validates correct usage
\end{itemize}


\newpage

\section{Blog}

Here I make various quick notes, partly for my own benefit so that I 
can remember why I did something, or to record things for the future, 
and partly to inform others about the reasons for my actions.

The thoughts are not carefully organised or well explained -- please 
ask questions if you think something looks interesting but unclear.

\begin{itemize}

\item
In designing the parallel loop syntax I wanted something simple and 
clean, without lots of redundant information.  This is how I designed
the initial CPU implementation, but the more I think about it and 
how the source code transformation will turn it into optimised code, 
the more I think it's better to go with something closer to the original
OPlus syntax with redundant information.  This will enable compile-time 
optimisations (e.g. unrolling of loops copying data into local 
shared-memory arrays), and the redundant info can be checked 
for consistency.

\item
I'm not sure how best to handle different datatypes.  My initial 
implementation supported only datasets composed of floats.  Rather than 
having different dataset types for each basic type, I think it may be 
best to change the struct to use void pointers and store the underlying 
datatype. Alternatively, can a C++ guru construct a parameterised class?
If I use void pointers, then I should do run-time type-checking.

\item
I'm going to have to be careful about the total size of all of the 
execution plans, but I think they won't be too large compared to the 
datasets.

\item
The optimal block sizes may be hard to determine a priori.  It may make
sense to use dynamic optimisation, adjusting them at run-time to reduce
execution time.

\item
When pointers are not used, data can be loaded straight into registers,
not shared-memory.

\item
When pointers are used, it's best to use shared-memory rather than relying 
on caching because with shared-memory only need to store the data which 
is needed.

%\item
%I don't know whether it will be worth using texture memory for read-only 
%accesses on the new GPUs -- possibly not because of the long latency even 
%when the data is in the texture cache.

\item
In the future, the labels for various routines can be handled automatically 
either through using macros or through the source code transformation.

\item
There are alternatives to the thread-colouring approach.  One is to use 
atomic locks, and another is to store the increments in shared-memory 
without combination then use one thread for each output to combine things
sequentially.  Ideally, it would be good to try each of these alternatives.

\item
Coping with a variable number of parameters might be painful.  C++ has 
{\tt va\_list} to handle a variable number of inputs, but unlike MATLAB 
I think there's no support for calling a function with a variable number 
of parameters.  FORTRAN can't do either which is why OPlus had a separate 
call for each argument, which coincidentally is similar to what happens in 
OpenCL.  Maybe I should stick with that approach for FORTRAN, and for C++ 
use a template with the number of parameters being specified as a template 
parameter?

\item
The {\tt m} loops in {\tt op\_par\_loop} would be unrolled by the source code 
transformation, and lots of code eliminated by evaluating many of the {\tt if} 
tests.  The NVIDIA compiler would then automatically unroll the little 
{\tt p} loops in most cases.


\item
Should investigate Trilinos as an alternative to ParMetis for partitioning.

\item
HDF5 is now the standard underneath CGNS.

\end{itemize}


\newpage

\section{Preprocessor}

I have developed a prototype preprocessor in MATLAB.  This parses 
the input source code, looking for all {\tt op\_par\_loop} calls.
It then parses their arguments and based on this it generates a CUDA
kernel routine which will execute on the GPU, and a host stub code 
which will select the execution plan and call the kernel routine.
Each of these aspects will now be described in some detail.

\subsection{Argument parsing}

The arguments are parsed to identify which datasets are addressed 
indirectly, using pointers, and which are addressed directly.

Those which are addressed indirectly are further analysed to see
if the same indirect dataset is used for more than one argument
with the same access type (e.g.~read-only, or increment) but
perhaps a different pointer (or a different index within a 
vector pointer table).  For example, in the HYDRA CFD code, 
a flux calculation for a single edge has 2 arguments which 
correspond to the flow variables at either end of the edge. 
It is the same flow variable dataset in each case, but the 
first edge-node pointer index is used for the first node, and 
the second is used for the second node.

Identifying shared indirect datasets is important to minimise
the use of shared memory within the GPU.

At the end of the parsing process, the preprocessor has a 
list of all indirect datasets and their access types, and a 
list of all of the main arguments and their access types and 
also their mapping to indirect datasets where relevant.

There is an important assumption that if the same dataset is
used for different arguments with different access types, then
there is no data conflict between the two.  Therefore, each will
be treated by the parser and the subsequent code generated by 
the preprocessor as a distinct indirect dataset.

\subsection{Kernel code generation}

If there are no indirect datasets, the kernel code which is 
generated is relatively simple.  Each thread block executes
one section of the set.  Read-only data is read from global 
memory into local registers, and increment arguments are 
initialised to zero.  A suitably-wrapped copy of the user's 
kernel function is then invoked as an inlined routine.  
Write-only data is written back to the global memory, and
for increment data the values held in global memory are 
incremented.


If there are indirect datasets then the kernel code is 
significantly more complicated.  Each thread block again
executes one section of the set, but now shared memory
is used to hold the indirect datasets.  The first step 
is to read in the appropriate read-only indirect datasets 
and initialise to zero the increment indirect datasets.
Next comes the execution phase, in which the indirect 
dataset values are taken from the shared memory arrays.
Thread coloring is used during the incrementing of the 
indirect dataset values, to prevent data conflict.
Finally, after the completion of the execution phase
the write-only indirect datasets are written to global 
memory, and the global memory values for the increment 
indirect datasets are incremented accordingly.

For further details it is best to read the code which 
is generated complete with comments describing each 
phase of its execution.

\subsection{Stub code generation}

If there are no indirect datasets, the stub code
simply calls the kernel routine once with the required
number of thread blocks.

If there are indirect datasets, the stub code defines
the mapping from arguments to indirect datasets and then
calls a library routine which either calculates an 
appropriate execution plan, or identifies one which has
been generated previously using the same inout arguments.
The stub code then calls the kernel routine several times,
once for each block color required by the execution plan
to avoid data conflicts between the blocks.

\subsection{New main code generation}

The final task for the preprocessor is to create a new main 
code.  This involves changing the header file which is used,
defining a number of routine prototypes, and changing slightly
the syntax of the {\tt op\_par\_loop} calls.

\subsection{Build process}

Figure \ref{fig:seq} shows the build process for a single 
thread CPU executable.  The user's main program (in this case 
{\tt jac.cpp}) uses the OP header file {\tt op\_seq.h} and is 
linked to the OP routines in {\tt op\_seq.c} using {\tt g++},
perhaps controlled by a Makefile.

Figure \ref{fig:op} shows the build process for the corresponding
CUDA executable.  The preprocessor parses the user's main program 
amd produces a modified main program and a CUDA file for 
each of the kernel functions.  These are then all compiled and 
linked to the OP routines in {\tt op\_lib.cu} using {\tt g++}
and the NVIDIA CUDA compiler {\tt nvcc}, again perhaps controlled 
by a Makefile.

\newpage

\begin{figure}
\begin{center}
{\setlength{\unitlength}{1in}
\begin{picture}(4.5,2)

\put(-0.2,1.6){\framebox(0.8,0.3){\tt op\_seq.h}}
\put(1,1.5){\framebox(1,0.5){\tt jac.cpp}}

\put(0.65,1.75){\line(1,0){0.1}}
\put(0.85,1.75){\vector(1,0){0.1}}

\put(2.5,1.5){\framebox(1,0.5){\tt op\_seq.c}}

\put(1.5,1.5){\vector(0,-1){0.625}}

\put(3,1.5){\vector(0,-1){0.625}}

\put(2.25,0.5){\oval(2.5,0.75)}
\put(2.25,0.5){\makebox(0,0){make / g++}}

\end{picture}}
\end{center}

\caption{Sequential code build process}
\label{fig:seq}
\end{figure}



\begin{figure}
\begin{center}
{\setlength{\unitlength}{1in}
\begin{picture}(7,5)

\put(1.5,4){\framebox(1,0.5){\tt jac.cpp}}

\put(2.0,4.0){\vector(0,-1){0.625}}

\put(2.1,3.0){\oval(4.2,0.75)}
\put(2.1,3.0){\makebox(0,0){op2.m preprocessor}}

\put(0.5,2.625){\vector(0,-1){0.625}}
\put(2.0,2.625){\vector(0,-1){0.625}}
\put(3.5,2.625){\vector(0,-1){0.625}}

\put(0.0,1.5){\framebox(1,0.5){\tt jac\_op.cpp}}
\put(1.3,1.5){\framebox(1.2,0.5){\tt res\_kernel.cu}}
\put(2.8,1.5){\framebox(1.4,0.5){\tt update\_kernel.cu}}
\put(4.5,1.5){\framebox(1,0.5){\tt op\_lib.cu}}

\put(0.5,1.5){\vector(0,-1){0.625}}
\put(2.0,1.5){\vector(0,-1){0.625}}
\put(3.5,1.5){\vector(0,-1){0.625}}
\put(5.0,1.5){\vector(0,-1){0.625}}

\put(2.75,0.5){\oval(5.5,0.75)}
\put(2.75,0.5){\makebox(0,0){make / nvcc / g++}}

\end{picture}}
\end{center}

\caption{CUDA code build process}
\label{fig:op}
\end{figure}


\clearpage

\section{Atomic operations and global reductions}

As illustrated in figure \ref{fig:atomic1}, when different threads 
try to update the same variable at the same time, there is a danger
that the resulting answer will be incorrect.  In the scenario in the
figure, the final value includes the increment from thread 1, but 
not the increment from thread 0. This problem is avoided by using an 
atomic add which makes the combined read/add/write a single operation, 
so that the atomic add by thread 0 is guaranteed to have completed 
before the start of the atomic add by thread 1.




\begin{figure}
\begin{center}
{\setlength{\unitlength}{0.4in}
\begin{picture}(10,5)

\thicklines

\put(0,4.5){\vector(0,-1){4.5}}
\put(-0.1,2.5){\makebox(0,0)[r]{time}}

\put(2.5,4.5){\makebox(0,0){\underline{without atomics}}}
\put(7.5,4.5){\makebox(0,0){\underline{with atomics}}}

\put(1.5,4.0){\makebox(0,0){thread 0}}
\put(3.5,4.0){\makebox(0,0){thread 1}}

\put(1.5,3.2){\makebox(0,0){read}}
\put(1.5,2.2){\makebox(0,0){add}}
\put(1.5,1.2){\makebox(0,0){write}}

\put(3.5,2.5){\makebox(0,0){read}}
\put(3.5,1.5){\makebox(0,0){add}}
\put(3.5,0.5){\makebox(0,0){write}}


\put(6.5,4.0){\makebox(0,0){thread 0}}
\put(8.5,4.0){\makebox(0,0){thread 1}}

%\put(6.5,3.0){\makebox(0,0){read/add/write}}
%\put(8.5,1.5){\makebox(0,0){read/add/write}}
\put(6.5,3.0){\makebox(0,0){atomic add}}
\put(8.5,2.0){\makebox(0,0){atomic add}}

\end{picture}}
\end{center}
\caption{Addition with and without atomic execution}
\label{fig:atomic1}
\end{figure}


Atomic operations are supported by the NVIDIA hardware and CUDA, 
but only for integer operations.  However, there is a special atomic 
compare-and-swap instruction {\tt atomicCAS} which can be used
for a software implementation of atomic floating point operations.

The syntax for {\tt atomicCAS} is:

{\tt int atomicCAS(int* old, int compare, int new);}

\noindent
If {\tt old} matches {\tt compare}, the value of {\tt old} is changed
to {\tt new}; otherwise it is left unchanged.  In either case, the
function returns the initial value of {\tt old}.
Table \ref{tab:atomic} shows how this can be used to implement 
an atomic add.  The \verb!float_as_int! and \verb!int_as_float!
operations are type casts which do not change the bit-wise values.
The key idea is that the routine computes a new value, but this is 
only stored if the old value hasn't changed in the meantime.
If it has, the routine has to repeat the operation until
it successfully updates the old value.

This example is easily modified to perform other reduction 
operations such as computing a minimum or a maximum.  This 
will be useful in performing global reduction operations
such as max / min / sum.  Each thread in a block can first 
do the reduction for the set elements it processes.  The 
values for all of the threads in the block can then be 
combined using the techniques outlined in the CUDA SDK 
{\tt reduction} example, and then the global reduction 
value can be obtained by thread 0 of eack block performing
an atomic update of a single variable in the global memory 
space.  On the Fermi GPU, the global variable will be held 
in the shared L2 cache and so the {\tt atomicCAS} operation 
will be performed efficiently with minimal latency.

\begin{table}
\caption{Implementation of floating point atomic addition
        (code supplied by Jonathan Cohen of NVIDIA Research)}
\label{tab:atomic}
\begin{verbatim}
float atomicAdd(float *addr, float val) {
  float old = *addr, old2;

  do {
       old2 = old;
       old  = int_as_float(atomicCAS( (int*)addr,
                                      float_as_int(old2),
                                      float_as_int(old2+val) ));
  } while( old2 != old );

  return old;
}
\end{verbatim}
\end{table}

\newpage

\section{Register pressure}

One concern raised by Jamil Appa, based on his experience with 
porting CFD codes to CUDA, is that each thread can have only a very 
limited of registers, 128 on the GT200 series and only 64 on Fermi
(even though each core has 2048 registers on the GT200 and 1024 
on Fermi).

If more than this number are needed then they ``spill over'' into 
so-called ``local memory'' which resides in the main global memory 
with all of the limitations in latency and bandwidth that entails.
With Fermi, this is mitigated only slightly through the use of the 
16kB L1 cache, corresponding to 2k DP variables.  If there are 
128 active threads, this corresponds to just 16 DP variables per
thread, so this isn't much of an addition to the register set.

To cope with this, we may need to change the way in which some kernels 
are implemented.  At present, for each element in the active set, 
the thread copies the required input argument values into registers
and then calls the user-supplied element function.  For the viscous 
edge calculation in HYDRA, this would require two $7\!\times\! 4$
``gradient'' arrays, plus four vectors of size $7$ for the flow 
variables and residuals at the two nodes, and hence at least 84 
DP variables, neglecting the registers required to perform the 
calculations.

For read-only arguments (such as the gradient arrays in the viscous 
edge calculation) it would be better to simply pass in the pointer
to the shared-memory location.

For increment arguments, using shared-memory pointers would raise 
the problem of data dependencies.  When using thread coloring, it 
would be necessary to wrap it around the entire element calculation,
rather than simply the incrementing operation as at present. Because 
of this, it might be best to re-order the set elements to minimise
the number of different colors within each warp.  Doing this would
disrupt the identity mapping for direct data references, which means
an additional mapping would need to be stored and fetched.

Alternatively, the use of atomic updates would either require code 
transformation of the user-supplied element function, or the user 
would have to use a special function to perform the increment and 
this would then be mapped to the required atomic operations.

\newpage

\section{Auto-tuning}

The more I get into the details of implementation, the more I see
the possibility for different methods of implementation, as well
as uncertainty in the optimal values for various parameters which
need to be specified.   To optimise the run-time performance, I 
think it will be essential to use auto-tuning techniques, building 
on ideas already used by packages such as ATLAS and FFTW.  To
emphasise this point, I don't think I am talking about squeezing 
out an extra 10\% of performance; I think it is more likely that
auto-tuning will double the performance.


Here I list some of the choices to be optimised, and the tradeoffs
they involve:
\begin{itemize}
\item
Size of block partition

To maximise date reuse within eack block, the natural choice is 
to make the block partition as large as possible subject to the
constraint that it fits inside the shared memory.  However, 
making it a bit smaller would allow multiple blocks to run 
concurrently on the same multiprocessor, and the run-time 
scheduler would then naturally overlap I/O for one block
with computation for another.

\item
Number of threads per block

One option is to have one thread per set element, giving the 
run-time scheduler the maximum freedom.  However, this increases
the cost of thread synchronisation, might run into difficulties 
with the total register usage, and doesn't amortise startup 
costs over multiple elements.

\item
Thread coloring or atomic increments

A tradeoff between synchronisation costs and idle threads, 
versus the cost of hardware-assisted atomic operations.

\item
Use of registers or shared-memory pointers

This was explained in the previous section
\end{itemize}


These choices exist for each parallel loop.  What is optimal 
for one loop may not be optimal for another.

Optimisation strategies which could be used include:
\begin{itemize}
\item
Brute force

Specify a small set of possible values for the integer variables
(block partition size and number of threads per block) and then 
systematically try every combination of choices and values.

This would be particularly applicable if, perhaps through 
profiling, the user can identify the most important loop(s).

\item
Genetic algorithms (or other stochastic method)

Brute force could work at the level of a single loop.  To handle
lots of loops, one either uses brute force on each one, one at 
a time, or one could perhaps use a higher-dimensional method
like GA to optimise them all at the same time.

\item
Run-time optimisation

In some cases, the optimal choice/value may depend on the specifics 
of the problem being solved, i.e.~may depend on the run-time data.
If an application requires many timesteps or iterations, it could 
be feasible at the beginning to try different choices/values for
different timesteps/iterations and monitor the execution times to
decide on the best combination.

\item
Machine learning

Maybe Mike O'Boyle could use his machine learning approach to 
anticipate the best choice for parallel loop each based on its 
defining characteristics.

\end{itemize}



\newpage

\section{OpenMP and SSE/AVX implementation}


Intel's Nehalem CPUs have a separate on-chip 256KB L2 cache for each 
core, and seem to have at least 200GB/s bandwidth from the L2 cache 
to the core.

Because of this, I think the simplest approach to an OpenMP 
implementation is to follow a very similar approach to the GPU 
implementation but with a single thread for each mini-partition. 
That way, there's no data conflict within a mini-partition, and 
the OpenMP parallelism would apply to the loop over all of the 
mini-partitions of a particular colour. 
The implementation would be simpler than the CUDA implementation 
since no global$\rightarrow$local renumbering would be required, 
or any of the low-level thread colouring. 

A similar strategy could be used for SSE/AVX vectorization as well, 
with each SSE/AVX vector element handling a different mini-partition. 
The mini-partitions would need to be reduced in size so that the 
required number (up to 8 for single precision computation in AVX) 
can fit into the 256KB L2 cache.

There is a potentially irritating implementation detail concerning the 
fact that the set size will usually not be an exact multiple of the number 
of threads times the length of the SSE/AVX vectors.  By using recursive 
bisection, the number of partitions can be made a power of 2, and their 
size made equal to within a maximum discrepancy of 1.  I think it should 
be possible to colour the mini-partitions so that the number of 
mini-partitions of each colour is a multiple of the length of the SSE/AVX 
vectors.  The only problem then is that the final SSE/AVX vector for
each mini-partition may not be complete, due to the difference in size of
the mini-partitions.  I think this can be dealt with by replicating one 
of the final elements to fill the empty bits of the vector.  The redundant 
computations will not affect the final outcome.  The same trick can be 
used to replicate an entire mini-partition if the number of mini-partitions 
of one colour is not a multiple of the length of the SSE/AVX vectors. 

This should result in very clean, efficient OpenMP SSE/AVX code.  The 
complexity will all be hidden in the plan construction.



\end{document}






\section{OPlus2 team}

Tentatively, these are the groups/people who might work on this project.

OPlus2 development:
\begin{itemize}
\item
Oxford -- 
\href{http://www.comlab.ox.ac.uk/mike.giles/}
     {Mike Giles}
and Gerd Heber

I was responsible for the original design of OPlus, although my 
post-doc Paul Crumpton did almost all of the original coding.  
I do have experience of CUDA programming for a financial Monte 
Carlo application, and as the original lead programmer for HYDRA 
I understand its requirements.

Gerd Heber is a member of the Oxford-Man Institute of Quantitative 
Finance assisting a number of computational projects.  He has just 
arrived from the Cornell Theory Center (Cornell's supercomputing centre)
and is very experienced in parallel software development and software 
engineering.

\item
Cambridge -- 
\href{http://www.trinhall.cam.ac.uk/about/contact_directory_profile.asp?ItemID=573}
     {Graham Pullan}
and Tobias Brandvik

Tobias has implemented a CFD code on an ATI card using the Brook GPU 
language, and is now gaining experience with CUDA.  He is starting
a PhD at Cambridge in the Whittle Laboratory on CFD implementation on 
novel multicore graphics hardware.

Graham Pullan is Tobias' supervisor, and a Rolls-Royce Research Fellow 
in the Whittle Laboratory.

\item
Queen's University Belfast -- 
\href{http://portal.surrey.ac.uk/eng/staff/staffac/HillsN}
     {Christophe Denis}
(and Stan Scott?)

Christophe Denis is an Associate Professor at University Pierre et Marie 
Curie (Paris 6) in the Department of Scientific Computing (LIP6).  He is 
spending the next 18 months as a Research Fellow in the Dept.~of Electrical 
Engineering and Computer Science at QUB, working with Prof Stan Scott.


\item
Surrey -- 
\href{http://www-anp.lip6.fr/~denis/index.html}
     {Nick Hills}

Nick Hills is an Academic Fellow in the Fluids and Systems group.  
Nick has extensive parallel computing experience and knows more about 
the internals of OPlus than anyone else.  However, it's not clear he will 
have time to contribute to this project, at least in its initial stages.

\item
NVIDIA, University of Illinois and/or CFD group at Stanford

These groups might be interested in participating, but probably not in the 
initial stages -- may be helpful later for profiling, code optimisation
and CFD applications.  NVIDIA might be willing to help at an early stage
by reviewing our technical plans and providing constructive criticism.

\end{itemize}


Application codes using OPlus2
\begin{itemize}
\item
Rolls-Royce/Surrey -- Leigh Lapworth, Nick Hills and others

HYDRA CFD code

\item
Nottingham -- 
\href{http://www.maths.nottingham.ac.uk/personal/ph/}
     {Paul Houston}
and Edward Hall

DGFEM (Discontinuous Galerkin Finite Element Method) 

\item
Oxford -- 
\href{http://www.comlab.ox.ac.uk/endre.suli/}
     {Endre Suli}
and David Knezevic

Materials FEM code

\item
BAESystems -- Jamil Appa

FLITE-3D, an electromagnetic code and/or the Solar suite

\item
Stanford?

CFD code
\end{itemize}

It would be good to have at least 3 application codes, particularly
if we decide to ask for EPSRC research funding, and I think CFD, 
materials modelling and computational electromagnetics are a good
representative set of applications.  Another possibility might be
oil reservoir modelling with Bath and/or Schlumberger (where Paul
Crumpton now works).



\section{Pre-processor}

In OPlus, we designed things so we simply had to link in the 
correct library to get either sequential or parallel executables.
This was partly because we lacked the knowledge to do anything
more sophisticated, such as using a pre-processor to make
modifications to the code before compilation.

This time, my starting point is to design the library interface
to be the best it can be from the point of view of the user, and 
therefore I think we need to have a pre-processor to prepare the 
code for compilation.

The pre-processor will need to carry out two functions.  The
first is to convert the specification of the parallel loops
into the appropriate lines of code to load and later store the 
required data from device memory into local shared storage.
The second is to convert the specification of the header file
or include file on the host into the appropriate specification
of the global constant data on the device, and perform the 
necessary copying across from host to device.

Given the desire to end up with a package which will run 
equally happily under both Unix/Linux and Windows, it may be 
that this pre-processor which I will later refer to as ``cpp'' 
will actually be a Java-based code.



\section{CUDA performance}

One of the keys to high performance using CUDA is to achieve
``coalescence'' of memory transfers between the main device memory 
and the registers and local shared memory of the GPU.  This 
improves the memory bandwidth by up to a factor 10 compared to
irregular memory accesses, and is identified in NVIDIA presentations
as a prime concern for code optimization.

Achieving ``coalescence'' with a warp (or half-warp) of 32 (or 16) 
threads requires that they load in a contiguous block of memory.
This is not natural when using unstructured grids with irregular 
patterns of memory access, but it could be achieved by grouping 
all set elements into larger ``super-elements''.  For example, 
the nodes of a computational grid could be grouped in fours into 
``super-nodes'' (perhaps by using recursive geomteric bisection
with the lower ``half'' rounded up in size to the nearest multiple 
of 4).  In CFD applications, there are often 6 or 7 floats per node,
so the combined data for a single ``super-node'' could be loaded 
in one coalesced warp transfer.


\section{Parallelisation strategies}

\subsection*{Strategy 1}

The first strategy is to not use the shared memory and instead 
go for the simplest possible approach in which each thread handles 
precisely one element of each set, loading in the required data
from the main device memory, and then writing the modified data
back to the device memory afterwards.  

The only difficulty with this approach is when there is an update 
to data through a pointer.  For example, when operations on an 
edge modify the data at the nodes at either end of the edge then 
there is a data conflict if two edges try to update the same node 
at the same time.  

One approach to this involves the use of atomic updates, but 
currently this is only supported in CUDA for integer data, and 
implementing it in software for other data types (using the 
``supermarket cheese counter'' approach of taking a ``number'' 
through an atomic update and then processing in turns) is likely 
to be inefficient.

I think the efficient may of handling it is likely to be through
the standard approach of ``colouring'' the set elements so that 
no two elements of the same colour update the same data.  For
a typical unstructured grid I would expect 10-20 colours to
be required.  The code executes each colour in parallel, then 
synchronizes before doing the next colour.

This strategy is very simple to implement, but its drawback
is that it leads to very high bandwidth requirements between 
the device main memory and the registers.

\subsection*{Strategy 2}

The second strategy reduces the bandwidth requirements by
loading in the data needed by a block of elements into
the local shared memory.  Multiple threads then work on this 
data before eventually sending back the updated values to 
the main memory.

If each edge has two end nodes, and there are 5 edges per node
on average, then when working on edges which point to nodes
strategy 1 requires the loading of 2 nodes per edge, whereas 
strategy 2 requires the loading of 0.2 nodes per edge, a bandwidth 
savings of factor 10.  Because the blocks can't be too large 
because of the limited size of the local shared memory we may 
lose a factor 2 compared to ideal but I still think it will 
much better this way as I expect the overall performance to
be memory-bandwidth limited in many cases.

Blocks will need to be coloured to avoid conflicts between blocks,
and threads will need to be coloured within blocks for the same 
reason.



\subsection{Super-elements and blocking}

Super-elements can be constructed through a recursive bisection 
procedure.  Given $d$-dimensional ``coordinates'' for each element,
the bisection can cycle through each dimension in turn.  At each 
level, a standard bisection which split a tree node of size $S$
into two parts, the larger being of size $S/2$ rounded up to 
the nearest integer.  By changing this for $S>32$ (32 being the 
warp size) to round up to the nearest multiple of 32, and also 
padding the original set size to a multiple of 32, we end up
with each element being a member of tree-nodes/super-elements 
of size 1, 2, 4, 8, 16 and 32.  This allows the super-element 
size to be chosen dynamically as part of the blocking ``plan''
depending on the width of data required by the parallel loop.
i.e. if there is just one float per element, then a super-element 
size of 32 would be good, whereas if there are 32 floats per element 
then working with single elements is fine from the point of view 
of memory coalescence.

If the binary tree information is stored, it can be re-used for 
blocking by using the largest tree nodes whose memory requirements 
fit within the limited shared memory size. i.e a tree node is used as 
a block if it fits within the shared memory and its parent does not.
(Needs a figure here to make this clear.)





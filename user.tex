%
%--  Text width, line spacing, etc.
%
 \documentstyle[11pt]{article}
 \topmargin 0.in  \headheight 0pt  \headsep 0pt  \raggedbottom
 \oddsidemargin 0.1in 
 \textheight 9.5in  \textwidth 6.00in 
 \parskip 5pt plus 1pt minus 1pt
 \def \baselinestretch {1.25}   % one-and-a-half spaced
 \setlength {\unitlength} {0.75in}
%
%
\newenvironment{routine}[2]
{\vspace{.25in}{\noindent\bf\hspace{0pt} #1}{\\ \noindent #2}
\begin{list}{}{
\renewcommand{\makelabel}[1]{{\tt ##1} \hfil} 
\itemsep 0pt plus 1pt minus 1pt
\leftmargin  1.2in
\rightmargin 0.0in
\labelwidth  1.1in
\itemindent  0.0in
\listparindent  0.0in
\labelsep    0.05in}
}{\end{list}}
%

\begin{document}

\title{OP2 User's Manual (phase 1)}
\author{Mike Giles}
\maketitle


\newpage
\section{Introduction}

OP2 is an API with associated libraries and preprocessors to 
generate parallel executables for applications on unstructured grids.  
The initial API is for C, but FORTRAN 77 will also be supported.

The key concept behind OP2 is that unstructured grids can be described
by a number of sets.  Depending on the application, these sets might
be of nodes, edges, triangular faces, quadrilateral faces, cells of
a variety of types, far-field boundary nodes, wall boundary faces, etc.
Associated with these sets are both data (e.g.~coordinate data at 
nodes) and pointers to other sets (e.g.~edge pointers to the two 
nodes at each end of the edge).  All of the numerically-intensive
operations can then be described as a loop over all members of a set, 
carrying out some operations on data associated directly
with the set or with another set through a pointer.   

OP2 makes the important restriction that the order in which the
function is applied to the members of the set must not affect the
final result.  This allows the parallel implementation to choose
its own ordering to achieve maximum parallel efficiency.

Two other restrictions in the current implementation are that 
the sets and pointers are static (i.e.~they do not change) and the 
operands in the set operations are not referenced through a double 
level of pointer indirection (i.e.~through a pointer to another set 
which in turn uses another pointer to data in a third set).

In the long-term, this library will allow users to write a single
program which can be built into a variety of different executables for
different platforms:
\begin{itemize}
\item
single-threaded on a CPU
\item
multi-threaded / vectorised on a CPU using OpenMP and/or SSE/AVX vectors
\item
parallelised on GPUs using CUDA or OpenCL
\item
distributed-memory MPI parallelisation in combination with any of the above
\end{itemize}

The higher-level distributed-memory MPI parallelisation will 
require parallel file I/O and so there will be routines to handle 
file I/O for the main datasets, as well as routines to handle 
terminal I/O.

However, the initial version described in this document is for 
execution on a shared-memory system, and the initial implementation
is for CUDA execution on a single GPU.

\newpage
\section{Overview}

A computational project can be viewed as involving three steps:
\begin{itemize}
\item
writing the program
\item
debugging the program, often using a small testcase
\item
running the program on increasingly large applications
\end{itemize}

With OP2 we want to simplify the first two tasks, while providing
as much performance as possible for the third one.  

To achieve the high performance for large applications, a 
preprocessor will be needed to generate the executable for GPUs.
However, to keep the initial development simple, the single-threaded 
executable does not use any special tools; the user's main code
is simply linked to a set of library routines, most of which do
little more than error-checking to assist the debugging process 
by checking the correctness of the user's program.  Note that this 
single-threaded version will not execute efficiently.  A new 
preprocessor (which has not yet been developed) will be needed 
to get efficient OpenMP/SSE/AVX execution on CPUs.

Figure \ref{fig:seq} shows the build process for a single 
thread CPU executable.  The user's main program (in this case 
{\tt jac.cpp}) uses the OP header file {\tt op\_seq.h} and is 
linked to the OP routines in {\tt op\_seq.c} using {\tt g++},
perhaps controlled by a Makefile.

Figure \ref{fig:op} shows the build process for the corresponding
CUDA executable.  The preprocessor parses the user's main program 
and produces a modified main program and a CUDA file which
includes a separate file for each of the kernel functions.  These 
are then compiled and linked to the OP routines in {\tt op\_lib.cu} 
using {\tt g++} and the NVIDIA CUDA compiler {\tt nvcc}, again 
perhaps controlled by a Makefile.

As well as the header file {\tt op\_seq.h} which is included by
the user's main code {\tt jac.cpp}, there is a header file
{\tt op\_datatypes.h} which is included by all of the files
in the CUDA implementation, and by {\tt op\_seq.h}.


In looking at the API specification, users may think it is
a little verbose in places. e.g.~users have to re-supply 
information about the datatype of the datasets being used
in a parallel loop.  This is a deliberate choice to simplify
the task of the preprocessor, and therefore hopefully reduce
the chance for errors.  It is also motivated by the thought that
{\bf ``programming is easy; it's the debugging which is difficult''}.
i.e.~writing code isn't time-consuming, it's correcting it
which takes the time.  Therefore, it's not unreasonable to ask
the programmer to supply redundant information, but be assured 
that the preprocessor or library will check that all redundant
information is self-consistent.  If you declare a dataset as being 
of type {\tt OP\_DOUBLE} and later say that it is of type 
{\tt OP\_FLOAT} this will be flagged up as an error at run-time, 
both in the single-threaded library and in the CUDA library.

\newpage

\begin{figure}
\begin{center}
{\setlength{\unitlength}{1in}
\begin{picture}(4.5,2)

\put(-0.2,1.6){\framebox(0.8,0.3){\tt op\_seq.h}}
\put(1,1.5){\framebox(1,0.5){\tt jac.cpp}}

\put(0.65,1.75){\line(1,0){0.1}}
\put(0.85,1.75){\vector(1,0){0.1}}

\put(2.5,1.5){\framebox(1,0.5){\tt op\_seq.c}}

\put(1.5,1.5){\vector(0,-1){0.625}}

\put(3,1.5){\vector(0,-1){0.625}}

\put(2.25,0.5){\oval(2.5,0.75)}
\put(2.25,0.5){\makebox(0,0){make / g++}}

\end{picture}}
\end{center}

\caption{Sequential code build process}
\label{fig:seq}
\end{figure}



\begin{figure}
\begin{center}
{\setlength{\unitlength}{1in}
\begin{picture}(7,5)

\put(1.5,4){\framebox(1,0.5){\tt jac.cpp}}

\put(2.0,4.0){\vector(0,-1){0.625}}

\put(2.1,3.0){\oval(4.2,0.75)}
\put(2.1,3.0){\makebox(0,0){op2.m preprocessor}}

\put(0.5,2.625){\vector(0,-1){0.625}}
\put(2.0,2.625){\vector(0,-1){0.625}}
\put(3.5,2.625){\vector(0,-1){0.625}}

\put(0.0,1.5){\framebox(1,0.5){\tt jac\_op.cpp}}
\put(1.3,1.5){\framebox(1.2,0.5){\tt jac\_kernels.cu}}
\put(2.8,1.5){\framebox(1.4,0.5){}}
\put(3.5,1.85){\makebox(0,0){\tt res\_kernel.cu}}
\put(3.5,1.65){\makebox(0,0){\tt update\_kernel.cu}}
\put(4.5,1.5){\framebox(1,0.5){\tt op\_lib.cu}}

\put(0.5,1.5){\vector(0,-1){0.625}}
\put(2.0,1.5){\vector(0,-1){0.625}}
\put(5.0,1.5){\vector(0,-1){0.625}}

\put(2.62,1.75){\vector(-1,0){0.1}}
\put(2.78,1.75){\line(-1,0){0.1}}


\put(2.75,0.5){\oval(5.5,0.75)}
\put(2.75,0.5){\makebox(0,0){make / nvcc / g++}}

\end{picture}}
\end{center}

\caption{CUDA code build process}
\label{fig:op}
\end{figure}


\clearpage

\newpage
\section{Initialisation and termination routines}

\begin{routine} {op\_init(int argc, char **argv)}
{This routine must be called before all other OP routines.}
\item \vspace{-0.4in}
\end{routine}

\begin{routine} {op\_decl\_set(int size, op\_set *set, char *name)}
{This routine declares information about a set.
All datatypes of the form {\tt op\_***} are defined in the header file.}

\item[size]          number of elements in the set
\item[set]           output OP set ID
\item[name]          a name used for output diagnostics
\item[ptr]           output OP pointer ID for identity mapping from set to itself
\item[ptrname]       a name for identity pointer used for output diagnostics
\end{routine}

\begin{routine} {op\_decl\_ptr(op\_set from, op\_set to, int dim, int *iptr, op\_ptr *ptr, char *name)}
{This routine declares information about a pointer from one set to another.}

\item[from]          set pointed from
\item[to]            set pointed to
\item[dim]           number of pointers per element
\item[iptr]          input pointer table
\item[ptr]           output OP pointer ID
\item[name]          a name used for output diagnostics
\end{routine}

\newpage

\begin{routine} {op\_decl\_dat(op\_set set, int dim, op\_datatype type, T *dat, op\_dat *data,\\ char *name)}
{This routine declares information about data associated with a set.}

\item[set]           set
\item[dim]           dimension of dataset (number of items per set element)
\item[type]          datatype ({\tt OP\_DOUBLE, OP\_FLOAT, OP\_INT, OP\_BOOL})
\item[idat]          input data of type {\tt T}  (checked for consistency with {\tt type} at run-time)
\item[dat]           output OP dataset ID
\item[name]          a name used for output diagnostics
\end{routine}

\begin{routine} {op\_decl\_const(int dim, op\_datatype type, T *dat, char *name)}
{This routine declares constant data with global scope to be used in user's kernel functions.
Note: in sequential version, it is the user's responsibility to define the appropriate global
variable.}

\item[dim]           dimension of data (i.e.~array size)
\item[type]          datatype ({\tt OP\_DOUBLE, OP\_FLOAT, OP\_INT, OP\_BOOL})
\item[idat]          input data of type {\tt T}  (checked for consistency with {\tt type} at run-time)
\item[name]          global name to be used in user's kernel functions;\\
                     a scalar variable if {\tt dim=1}, otherwise an array of size {\tt dim}
\end{routine}


\begin{routine} {op\_fetch\_data(op\_dat dat)}
{This routine transfers data from the GPU back to the CPU.}
\item[dat]           OP dataset ID -- data is put back into original input array 
\end{routine}


\begin{routine} {op\_diagnostic\_output()}
{This routine prints out various useful bits of diagnostic info about sets, pointers and datasets}
\item \vspace{-0.3in}
\end{routine}


\begin{routine} {op\_exit()}
{This routine must be called last to cleanly terminate the OP computation.}
\item \vspace{-0.3in}
\end{routine}



\newpage
\section{Parallel execution routine}

The parallel loop syntax when the user's kernel function has 3 arguments,
with the third being a local constant or global reduction array, is:

\begin{routine} {op\_par\_loop\_3(void (*kernel)(T0 *, T1 *, T2 *), char * name, op\_set set,\\
\hspace*{0.13in}    op\_dat arg0, int idx0, op\_ptr ptr0, int dim0, op\_datatype typ0, op\_access acc0,\\
\hspace*{0.13in}    op\_dat arg1, int idx1, op\_ptr ptr1, int dim1, op\_datatype typ1, op\_access acc1,\\
\hspace*{0.13in}    T2     *arg2, int idx2, op\_ptr ptr2, int dim2, op\_datatype typ2, op\_access acc2)}{}

\item[kernel]     user's kernel function with 3 arguments of arbitrary type\\
                  (this is only used for the single-threaded CPU build)
\item[name]       name of kernel function, used for output diagnostics
\item[set]        OP set ID, giving set over which the parallel computation is performed
\item[arg]        OP dataset ID, or pointer to constant or global reduction array
\item[idx]        index of pointer to be used (-1 $\equiv$ no pointer indirection)
\item[ptr]        OP pointer ID ({\tt OP\_ID} for identity mapping, i.e.~no pointer indirection,
                  {\tt OP\_GBL} for constant or global reduction array)
\item[dim]        dimension of dataset (redundant info, checked at run-time for consistency)
\item[typ]        datatype of dataset (redundant info, checked at run-time for consistency)
\item[acc]        access type:\\
                  {\tt OP\_READ}: read-only\\ 
                  {\tt OP\_WRITE}: write-only, but without potential data conflict\\ 
                  {\tt OP\_RW}:  read and write, but without potential data conflict\\
                  {\tt OP\_INC}: increment, or global reduction to compute a sum\\ 
                  {\tt OP\_MAX}: global reduction to compute a maximum \\ 
                  {\tt OP\_MIN}: global reduction to compute a minimum \\ 
\end{routine}

{\tt kernel} is a function with 3 arguments of arbitrary type which performs
a calculation for a single set element.
This will get converted by a preprocessor into a routine called by the CUDA kernel 
function.  The preprocessor will also take the specification of the arguments and turn 
this into the CUDA kernel function which loads in indirect data (i.e.~data addressed 
indirectly through pointers) from the device main memory into the shared storage, 
then calls the converted {\tt kernel} function for each element for each line in 
the above specification.  Indirect data is incremented in shared memory (with
thread coloring to avoid possible data conflicts) before being updated at the end 
of the CUDA kernel call.

The restriction that {\tt OP\_WRITE} and {\tt OP\_RW} access must not have any 
potential data conflict means that two different elements of the set cannot 
through pointer indirection reference the same elements of the dataset.  

%(If they did, it would be undetermined which final value would be assigned to
%the dataset element.  Users might argue that it would be OK if both assigned
%the same value, but I don't want to handle that complication.)

Different numbers of arguments are handled similarly by routines with names
of the form {\tt op\_par\_loop\_n} where {\tt n} is the number of arguments.
Each argument can be either a dataset or or constant or global reduction array,
following the syntax shown above.


\section{Preprocesor}

The prototype preprocessor has been written in MATLAB.  It is run by the command

{\tt op2('main')}

\noindent
where {\tt main.cpp} is the user's main program.  It produces as output a 
modified main program {\tt main\_op.cpp}, and a new CUDA file {\tt main\_kernels.cu}
which includes one or more files of the form {\tt xxx\_kernel.cu} containing 
the CUDA implementations of the user's kernel functions.

If the user's application is split over several files it is run by a command such as

{\tt op2('main','sub1','sub2','sub3')}

\noindent
where {\tt sub1.cpp, sub2.cpp, sub3.cpp} are the additional input files which will 
lead to the generation of output files {\tt sub1\_op.cpp, sub2\_op.cpp, sub3\_op.cpp} 
in addition to {\tt main\_op.cpp, main\_kernels.cu} and the individual kernel files.

The preprocessor cannot currently handle cases in which the same user kernel is 
used in more than one parallel loop, or when global constant data is set/updated 
in more than one place within the code.  This will be addressed in the future.


\section{Future changes}

There will be a new function {\tt \bf op\_partition} which will re-number 
all of the elements in each set, to maximise the data reuse within each 
mini-partition.  This is likely to use the same partitioning algorithm which
will be employed for the higher-level distributed-memory partitioning for the
MPI implementation in phase 2.

\end{document}




